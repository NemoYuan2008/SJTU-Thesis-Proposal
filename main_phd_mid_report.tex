%%%%%%%%%%%%%%%%%%%%%%%%%%%%%%%%%%%%%%%%%%%%%%%%%%%%%%%%%%%
% 这是博士生年度进展报告的 LaTeX 模板
% 如需开题报告模板,请使用 main.tex
% 使用时请仔细阅读注释并按照要求填写
%%%%%%%%%%%%%%%%%%%%%%%%%%%%%%%%%%%%%%%%%%%%%%%%%%%%%%%%%%%


\documentclass[a4paper,zihao=-4,AutoFakeBold]{ctexart}

\usepackage{geometry}
\usepackage{amsmath}
\usepackage{amssymb}
\usepackage{fancyhdr}
\usepackage{enumitem}
\usepackage{lastpage}
\usepackage{xcolor}
\usepackage{graphicx}
\usepackage{array}
\usepackage[style=gb7714-2015]{biblatex}
\addbibresource{ref.bib}
\usepackage{etoolbox}
\usepackage[hidelinks]{hyperref}

% 设置页边距
\geometry{
    top    = 2.5cm,
    bottom = 2.5cm,
    left   = 2.8cm,
    right  = 2.8cm,
}

% 使用 Times Roman 西文字体
\setmainfont{texgyretermes}[
    Extension      = .otf,
    UprightFont    = *-regular,
    BoldFont       = *-bold,
    ItalicFont     = *-italic,
    BoldItalicFont = *-bolditalic,
]

% 定义 ``研究课题来源'' 表格里用到的勾和框
\newcommand*{\myunchecked}{$\square$\kern.65pt}  % 未打勾的方框
\newcommand*{\mychecked}{\checkmark}   % 打勾

% 一级标题仿宋加粗、悬挂缩进
% 二级和三级标题楷书加粗,与正文字体一致
\ctexset{
    section = {
        name = {,、},     % 一级标题编号后的顿号
        aftername = {},   % 取消顿号后的间距
        format = \fangsong\bfseries\linespread{1.25}\selectfont,
        beforeskip = 0pt,
        afterskip = 0pt,
    },
    subsection = {
        format = \kaishu\bfseries,
        beforeskip = 0pt,
        afterskip = 0pt,
    },
    subsubsection = {
        format = \kaishu\bfseries,
        beforeskip = 0pt,
        afterskip = 0pt,
    },
}

\setlist{nosep}

\lhead{}
\rhead{}
\chead{\zihao{-5}\color{gray}上海交通大学博士生年度进展报告 Annual Progress Review for SJTU Doctoral Student}
\cfoot{\zihao{-5}\color{gray}\thepage\ / \pageref*{LastPage}}


\begin{document}

\pagestyle{empty}

\begin{figure}[h]
    \centering
    \includegraphics[width=10cm]{figures/sjtu-logo.png}
\end{figure}

\begin{center}
    \bfseries\songti
    \vspace{-0.5cm}
    \zihao{1} 博士研究生学位论文年度进展报告\par\vspace{18.8pt}
    \zihao{-3} Annual Progress Report for SJTU Doctoral Student
    \vspace{0.5cm}
\end{center}


%%%%%%%%%%%%%%%%%%%%%%%%%%%%%%%%%%%%%%%%%%%%%%%%%%%%%
% 填写说明:请在下面表格中填写你的信息,注意事项:
% 1. 入学方式 Enrollment 中应填写以下三个中的一个
%       a. 直博生 Doctoral Student after Bachelor's
%       b. 普博生 Regular Doctoral Student
%       c. 硕博连读生 Combined Master and Doctoral
% 2. 学习形式 Study Mode 中应填写以下两个中的一个
%       a. 学术型 Academic
%       b. 专业型 Professional
%%%%%%%%%%%%%%%%%%%%%%%%%%%%%%%%%%%%%%%%%%%%%%%%%%%%%
\begin{table}[h]
    \centering
    \renewcommand{\arraystretch}{1.7}
    % 左栏中文为楷书加粗四号字,英文为小四加粗;右栏为仿宋小四加下划线
    \zihao{-4}
    \begin{tabular}{>{\bfseries\kaishu}l>{\fangsong}m{9.3cm}}
        {\zihao{4}学号}~~Student ID
            & 012345678912\\\cline{2-2}
        {\zihao{4}姓名}~~Name
            & 交通大学\\\cline{2-2}
        {\zihao{4}导师}~~Supervisor(s)
            & 我的导师\\\cline{2-2}
        {\zihao{4}专业}~~Major
            & 我的专业\\\cline{2-2}
        {\zihao{4}学院}~~School
            & 我的学院\\\cline{2-2}
        {\zihao{4}入学方式}~~Enrollment
            & 直博生 Doctoral Student after Bachelor's\\\cline{2-2}
        {\zihao{4}学生类别}~~Degree Program
            & 学术型 Academic\\\cline{2-2}
        {\zihao{4}考核日期}~~Date
            & 20YY年MM月DD日\\\cline{2-2}
    \end{tabular}
\end{table}


\clearpage


\begin{center}
    \zihao{-3}
    \vspace*{0.5cm}
    {\heiti 填\quad 报\quad 说\quad 明}\par
    \vspace{0.66cm}
    {\bfseries Instruction}
\end{center}

\begin{enumerate}[parsep=.5\baselineskip]
    \fangsong
    \item 博士研究生年度进展报告应通过%
          \href{http://my.sjtu.edu.cn/}{\color{blue}\underline{数字交大}}%
          在线提交申请,填写本表并上传系统。
          特殊情况下经研究生院事先同意,可不上传系统,
          并使用《上海交通大学博士研究生年度进展报告评审表》完成评审。

          The application for thesis/dissertation work annual progress review should be submitted online through
          \href{http://my.sjtu.edu.cn/}{\color{blue}\underline{My SJTU}}.
          The student shall fill this form and upload it in the system.
          Under special circumstance, this form does not need to be
          uploaded and the review can be proceeded with the review
          form with prior consent from the graduate school.

    \item 本报告应A4纸双面打印,于左侧钉在一起。
          各栏空格不够时,请自行加页。考核前提前一周送交导师、评审专家审阅。

          This report should be printed with A4 papers and bound
          together on the left. If the space left is not enough,
          please feel free to add extra pages.
          The print version shall be sent to the supervisor,
          and the review committee members for review at least
          one week before the oral presentation.

    \item 年度进展报告通过后,定稿版报告由研究生、导师各存档一份,无需上传系统。
    
          Upon passing the review, 
          the final version of this report shall be archived by 
          the graduate student and his/her supervisors for 
          future reference.
\end{enumerate}

\clearpage


\pagestyle{fancy}
\setcounter{page}{1}


\begin{center}
    \fangsong\bfseries\zihao{4}
    博士生年度进展报告 Annual Progress Report
\end{center}

%%%%%%%%%%%%%%%%%%%%%%%%%%%%%%%%%%%%%%%%%%%%%%%%%%%%%%%%%%%
% 填写说明:请在下面表格中填写你的信息,注意事项:
% 1. 论文题目如果太长,使用 \newline 手动换行 (见示例)
% 2. 本模板提供了 \mychecked 命令用来打勾,
%    \myunchecked 命令用来画出未打勾的方框
%%%%%%%%%%%%%%%%%%%%%%%%%%%%%%%%%%%%%%%%%%%%%%%%%%%%%%%%%%%
\begin{table}[h]
    \centering
    \zihao{-4}\fangsong
    \linespread{1.68}\selectfont   % 与word仔细比对后选取的经验值
    \begin{tabular}{|m{3.3cm}|m{11cm}|}
        \hline
        论文题目\newline Proposed Title & 
        我的很长很长很长很长很长很长很长很长很长的\newline
        很厉害的论文题目\\
        \hline
        研究课题来源\newline Source of Research\newline Project &
        请在合适选项前画 \checkmark~~Please select proper options by ``\checkmark''.\newline
        \mychecked\ 国家自然科学基金课题 NSFC Research Grants\newline
        \myunchecked\ 国家社会科学基金 National Social Science Fund of China\newline
        \mychecked\ 国家重大科研专项 National Key Research Projects\newline
        \myunchecked\ 其它纵向科研课题 Other Governmental Research Grants\newline
        \myunchecked\ 企业横向课题 R\&D Projects from Industry\newline
        \myunchecked\ 自拟课题 Self-proposed Project
        \\\hline
        论文开题日期\newline Dissertation\newline Proposal Date&
        20YY年MM月DD日\\
        \hline
    \end{tabular}
\end{table}

\vspace{-12pt}

% 设置后续正文的字体:楷书、0.5倍段间距
% 行距倍数 1.75 为与官方 word 版仔细比对后选取的经验值
\kaishu
\setlength{\parskip}{0.5\baselineskip}
\linespread{1.75}\selectfont

% 设置浮动体内的字体为楷书小四号字,与正文一致
\makeatletter
\appto{\@floatboxreset}{\kaishu\zihao{-4}}
\makeatother


%%%%%%%%%%%%%%%%%%%%%%%%%%%%%%%%%%%%%%%%%%%%%%%%%%%%%%%%%%%%%
% 填写说明:开始撰写你的开题报告。
% 1. \section 里是官方开题报告模板里的填写要求,已为你准备好。
% 2. 请接在后面填写你的正文内容,具体请仔细阅读示例和注释。
% 3. 你可以使用 \subsection 和 \subsubsection 命令来使用小标题
%    所有小标题都可以使用 \label 打标签,并用 \ref 进行引用
%%%%%%%%%%%%%%%%%%%%%%%%%%%%%%%%%%%%%%%%%%%%%%%%%%%%%%%%%%%%%


\section{报告正文 Report。\mdseries 请阐述开题报告或上次年度进展报告以来学位论文研究工作的进展情况及所取得的阶段性成果,并简述下一年度研究计划,不少于3,000汉字。Please summarize your research progress and achievements since your dissertation proposal or last annual progress review, as well as your plan for next step. No less than 2,400 words if written in English.}

报告正文\cite{ZJSD}。

\vspace{2\baselineskip}% 让参考文献列表与正文有 2 行的间距
{%
    \linespread{1.25}\selectfont    % 参考文献行距小一点
    参考文献 References: 
    \printbibliography[heading=none]
}


\section{成果清单 List of Achievements。\mdseries 请列出开题报告以来或上次年度进展报告以来新发表的学术论文、授权专利、国际会议论文、专著等成果清单。作者、标题、杂志、卷、期、页码等信息请填写完整。Please provide a list of academic publications (papers, patents, international academic conference talks, monographs, etc.) since your dissertation proposal or last annual progress review. Information on author list, title, journal name, volume, number, and pages shall be complete.}

\begin{enumerate}[label={[\arabic*]}]
    \item 成果1
    \item 成果2
\end{enumerate}




% 剩下的所有内容均为仿宋加粗小四号字
\normalfont\zihao{-4}\bfseries\fangsong
\noindent
本人承诺:开题报告中的内容真实无误,若有不实,愿承担相应的责任和后果。
I hereby declare and confirm that the details 
provided in this Form are valid and accurate. 
If anything untruthful found, 
I will bear the corresponding liabilities and consequences.

\vspace{\baselineskip}



%%%%%%%%%%%%%%%%%%%%%%%%%%%%%%%%%%%%%%%%%%%%%%%%%%%%%%%%%%%%%
% 填写说明:记得填写姓名和日期
%%%%%%%%%%%%%%%%%%%%%%%%%%%%%%%%%%%%%%%%%%%%%%%%%%%%%%%%%%%%%
\noindent
学生签字/Signature of Student:交通大学
\hfill              
日期/Date: 20YY-MM-DD

\end{document}