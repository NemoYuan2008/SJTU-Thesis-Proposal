%%%%%%%%%%%%%%%%%%%%%%%%%%%%%%%%%%%%%%%%%%%%%%%%%%%%%%%%%%%
% 这是开题报告的 LaTeX 模板
% 如需其他模板,请阅读 README.md 文件中的说明
% 使用时请仔细阅读注释并按照要求填写
%%%%%%%%%%%%%%%%%%%%%%%%%%%%%%%%%%%%%%%%%%%%%%%%%%%%%%%%%%%

\documentclass[a4paper,zihao=-4,AutoFakeBold,fontset=windows]{ctexart}

\usepackage{geometry}
\usepackage{amsmath}
\usepackage{amssymb}
\usepackage{enumitem}
\usepackage{xcolor}
\usepackage{graphicx}
\usepackage{array}
\usepackage[style=gb7714-2015]{biblatex}
\addbibresource{ref.bib}
\usepackage{etoolbox}
\usepackage[hidelinks]{hyperref}

\geometry{
    top    = 2.5cm,
    bottom = 2.5cm,
    left   = 2.8cm,
    right  = 2.8cm,
}

\setmainfont{texgyretermes}[
    Extension      = .otf,
    UprightFont    = *-regular,
    BoldFont       = *-bold,
    ItalicFont     = *-italic,
    BoldItalicFont = *-bolditalic,
]

% 定义 ``研究课题来源'' 表格里用到的勾和框
\newcommand*{\myunchecked}{$\square$\kern.65pt}  % 未打勾的方框
\newcommand*{\mychecked}{\checkmark}   % 打勾

% 一级标题仿宋加粗、悬挂缩进
% 二级和三级标题楷书加粗,与正文字体一致
\ctexset{
    section = {
        name = {,、},     % 一级标题编号后的顿号
        aftername = {},   % 取消顿号后的间距
        format = \fangsong\bfseries\linespread{1.25}\selectfont,
        % 全设置为 0pt 是因为后面用了 \parskip
        beforeskip = 0pt,
        afterskip = 0pt,
    },
    subsection = {
        format = \kaishu\bfseries,
        beforeskip = 0pt,
        afterskip = 0pt,
    },
    subsubsection = {
        format = \kaishu\bfseries,
        beforeskip = 0pt,
        afterskip = 0pt,
    },
}

\setlist{nosep}


\begin{document}

\pagestyle{empty}

\begin{figure}[h]
    \centering
    \includegraphics[width=8.5cm]{figures/sjtu-logo.pdf}
\end{figure}

\begin{center}
    \bfseries\songti
    \vspace{-1.0cm}
    \zihao{1} 研究生学位论文开题报告\par\vspace{18.8pt}
    \zihao{-3} Graduate Thesis/Dissertation Proposal
    \vspace{0.5cm}
\end{center}


%%%%%%%%%%%%%%%%%%%%%%%%%%%%%%%%%%%%%%%%%%%%%%%%%%%%%
% 填写说明:请在下面表格中填写你的信息,注意事项:
% 1. 学生类别 Degree Program 中应填写以下四个中的一个
%       a. 学术型博士生 Academic Doctoral Student
%       b. 专业型博士生 Professional Doctoral Student
%       c. 学术型硕士生 Academic Master Student
%       d. 专业型硕士生 Professional Master Student
% 2. 学习形式 Study Mode 中应填写以下三个中的一个
%       a. 全日制 Full-time
%       b. 非全日制 Part-time
%       c. 同等学力学生
% 3. 论文题目如果太长,使用 \newline 手动换行 (见示例)
%%%%%%%%%%%%%%%%%%%%%%%%%%%%%%%%%%%%%%%%%%%%%%%%%%%%%
\begin{table}[h]
    \centering
    \renewcommand{\arraystretch}{1.7}
    % 左栏中文为楷书加粗四号字,英文为小四加粗;右栏为仿宋小四加下划线
    \zihao{-4}
    \begin{tabular}{>{\bfseries\kaishu}l>{\fangsong}m{9.3cm}}
        {\zihao{4}学号}~~Student ID
            & 012345678912\\\cline{2-2}
        {\zihao{4}姓名}~~Name
            & 我的姓名\\\cline{2-2}
        {\zihao{4}学生类别}~~Degree Program
            & 学术型博士生 Academic Doctoral Student\\\cline{2-2}
        {\zihao{4}学习形式}~~Study Mode
            & 全日制 Full-time\\\cline{2-2}
        {\zihao{4}导师}~~Supervisor(s)
            & 我的导师\\\cline{2-2}
        {\zihao{4}论文题目}~~Thesis Title
            & 我的很长很长很长很长很长很长很长很长很长的\newline
              很厉害的论文题目\\\cline{2-2}
        {\zihao{4}学院}~~School
            & 我的学院\\\cline{2-2}
        {\zihao{4}专业}~~Major
            & 我的专业\\\cline{2-2}
        {\zihao{4}开题日期}~~Date
            & 202Y-MM-DD\\\cline{2-2}
        {\zihao{4}开题地点}~~Venue
            & 会议室\\\cline{2-2}
    \end{tabular}
    \vspace{-5cm}   % 防止表格被排到下一页
\end{table}


\clearpage


\begin{center}
    \zihao{-3}
    {\heiti 填\quad 报\quad 说\quad 明}\par
    \vspace{0.66cm}
    {\bfseries Instruction}
\end{center}

\begin{enumerate}[parsep=.5\baselineskip]
    \fangsong
    \item 校本部研究生的开题报告应通过%
          \href{http://my.sjtu.edu.cn/}{\color{blue}\underline{数字交大}}%
          在线提交申请,填写本表并上传系统。
          特殊情况下经研究生院事先同意,可不上传系统,
          并使用《上海交通大学研究生论文开题评审表》完成评审。
          
          The application for thesis/dissertation proposal
          should be submitted online through
          \href{http://my.sjtu.edu.cn/}{\color{blue}\underline{My SJTU}}.
          The student shall fill this form and upload it in the system.
          Under special circumstance,
          this form does not need to be uploaded
          and the review can be proceeded with the review form
          with prior consent from the graduate school.

    \item 开题报告为A4大小,于左侧装订成册。各栏空格不够时,请自行加页。
          考核前提前一周送交导师、评审专家审阅。

          This form should be printed with A4 papers
          and bound together on the left.
          If the space left is not enough,
          please feel free to add extra pages.
          The print version shall be sent to the supervisor,
          and the review committee members for review
          at least one week before the oral presentation.

    \item 博士生导师可以根据博士生学位论文选题情况自行确定是否进行开题查新,
          博士学位论文开题查新报告应由查新工作站提供。

          The supervisor should decide, based on the proposed topics,
          whether a novelty assessment report is needed or not,
          which should be conducted by an authorized
          novelty assessment department.

    \item 开题报告通过后,定稿版开题报告由研究生、导师各存档一份,无需上传系统。

          Upon passing the proposal,
          the final version of this report shall be archived by the
          graduate student and his/her supervisors for future reference.

    \item 同等学力研究生开题答辩采用会议形式,
          硕士邀请至少3名相关学科/专业领域具有硕士研究生指导资格的专家。
          博士邀请5名相关学科/专业领域具有博士研究生指导资格的专家。

          The capstone presentation adopts a conference format,
          and at least three experts with master's degree guidance
          qualifications in relevant disciplines and professional fields
          are invited for the master's degree.
          And five experts with doctoral guidance qualifications
          in relevant disciplines/professional fields are invited
          for doctoral guidance.
\end{enumerate}

\clearpage


\pagestyle{plain}
\setcounter{page}{1}


%%%%%%%%%%%%%%%%%%%%%%%%%%%%%%%%%%%%%%%%%%%%%%%%%%%%%%%%%%%
% 填写说明:请在下面表格中填写你的信息,注意事项:
% 1. 论文题目如果太长,使用 \newline 手动换行 (见示例)
% 2. 本模板提供了 \mychecked 命令用来打勾,
%    \myunchecked 命令用来画出未打勾的方框
% 3. 若不勾选“其它”选项,请删除“在此处填写内容”,但不要删除包围它的
%    \hbox{} 代码,以保证表格及下划线格式正确
%%%%%%%%%%%%%%%%%%%%%%%%%%%%%%%%%%%%%%%%%%%%%%%%%%%%%%%%%%%
\begin{table}[h]
    \centering
    \zihao{-4}\fangsong
    \linespread{1.68}\selectfont   % 与word仔细比对后选取的经验值
    \begin{tabular}{|m{3.3cm}|m{11cm}|}
        \hline
        论文题目\newline Proposed Title & 
        我的很长很长很长很长很长很长很长很长很长的\newline
        很厉害的论文题目\\
        \hline
        研究课题来源\newline Source of Research\newline Project &
        请在合适选项前画 \checkmark~~Please select proper options by ``\checkmark''.\newline
        \mychecked\ 国家自然科学基金课题 NSFC Research Grants\newline
        \myunchecked\ 国家社会科学基金 National Social Science Fund of China\newline
        \mychecked\ 国家重大科研专项 National Key Research Projects\newline
        \myunchecked\ 其它纵向科研课题 Other Governmental Research Grants\newline
        \myunchecked\ 企业横向课题 R\&D Projects from Industry\newline
        \myunchecked\ 自拟课题 Self-proposed Project\newline
        \myunchecked\ 其它 Other\enspace
        \vtop{\hbox{在此处填写内容}\hbox{\rule[18pt]{8cm}{.25pt}}}\vspace{-12pt}
        \\\hline
    \end{tabular}
\end{table}


% 设置后续正文的字体:楷书、0.5倍段间距
% 行距倍数 1.75 为与官方 word 版仔细比对后选取的经验值
\kaishu
\setlength{\parskip}{0.5\baselineskip}
\linespread{1.75}\selectfont

% 设置浮动体内的字体为楷书小四号字,与正文一致
\makeatletter
\appto{\@floatboxreset}{\kaishu\zihao{-4}}
\makeatother


%%%%%%%%%%%%%%%%%%%%%%%%%%%%%%%%%%%%%%%%%%%%%%%%%%%%%%%%%%%%%
% 填写说明:开始撰写你的开题报告。
% 1. \section 里是官方开题报告模板里的填写要求,已为你准备好。
% 2. 请接在后面填写你的正文内容,具体请仔细阅读示例和注释。
% 3. 你可以使用 \subsection 和 \subsubsection 命令来使用小标题
%    所有小标题都可以使用 \label 打标签,并用 \ref 进行引用
%%%%%%%%%%%%%%%%%%%%%%%%%%%%%%%%%%%%%%%%%%%%%%%%%%%%%%%%%%%%%

\vspace{-\parskip}
\section{请综述课题国内外研究进展、现状、挑战与意义,可分节描述。
    博士生不少于 10,000 汉字,硕士生不少于 5,000 汉字。
    请在文中标注参考文献。 
    Please review the frontier, current status, 
    challenges and significance of the research topic. 
    The citations should be marked in the context 
    and listed in order at the end of this section. 
    No less than 8,000 words for doctoral students 
    and 4,000 words for master students if written in English.}


制作开题报告和中期报告 \LaTeX{} 模板的初衷是为了大家能
方便地输入数学公式 $c^2 = a^2 + b^2$、
进行文献引用\cite{ZJSD}、
进行格式排版(本模板提供了二级和三级标题的支持,这在官方 Word 模板中是没有的),
以及将来使用 \LaTeX{} 模板撰写学位大论文时能直接复用这些内容。


引用文献时,请将参考文献添加到项目根目录中的 \texttt{ref.bib} 文件中,
然后在正文中使用 \verb|\cite{}| 命令,例如\cite{SPDZ}。
这样引用是上标形式的,如果需要平行形式的引用,请使用
\verb|\parencite{}| 命令,效果如 \parencite{SPDZ}。


\subsection{二级标题}\label{sub}

你还可以使用二级标题和三级标题,
并使用 \LaTeX{} 的 \verb|\label| 和 \verb|\ref| 命令进行引用,
如第 \ref{sub} 节和第 \ref{subsub} 节所示。



\subsubsection{三级标题}\label{subsub}

最后要打印出参考文献列表,如下所示。


%%%%%%%%%%%%%%%%%%%%% 参考文献列表部分 %%%%%%%%%%%%%%%%%%%%%
\vspace{2\baselineskip}% 让参考文献列表与正文有 2 行的间距
{%
    \linespread{1.25}\selectfont    % 参考文献行距小一点
    \noindent 参考文献 References: 
    \printbibliography[heading=none]
}
%%%%%%%%%%%%%%%%%%%%% 参考文献列表结束 %%%%%%%%%%%%%%%%%%%%%


\section{课题研究目标、主要研究内容和拟解决的关键问题。 
    Research objectives, main contents and key issues to be solved.}

我的课题研究目标非常明确,主要研究内容非常的主要,拟解决的关键问题非常关键。



\section{拟采取的研究方法、研究方案及其可行性分析。
    Research methods and research scheme to be adopted 
    and feasibility analysis.}

我要采取特别厉害的研究方法、特别新颖的研究方案。这些方法和方案都非常的可行。



\section{课题的创新点 Novelties of the proposed topic.}

我的课题非常的创新,第一点是他的创新点很多,第二是每一个创新点都非常非常的创新。



\section{计划进度、预期成果 Research schedule, and expected outcomes.}

我计划一年内发各种顶级会议,然后再发 Nature 和 Science 这样的期刊,然后获得诺贝尔奖。
先获得诺贝尔计算机科学奖,然后是诺贝尔密码学奖,最后来个诺贝尔化学奖。



\section{与本课题有关的工作积累、已有的研究工作成绩。
    Prior experience and accomplished achievements 
    related to the proposed topic.}


目前已经积累了很多很多工作了,也取得了非常非常好的工作成绩。




\vspace{\baselineskip}

% 剩下的所有内容均为仿宋加粗小四号字
\normalfont\zihao{-4}\bfseries\fangsong
\noindent
本人承诺:开题报告中的内容真实无误,若有不实,愿承担相应的责任和后果。
I hereby declare and confirm that the details 
provided in this Form are valid and accurate. 
If anything untruthful found, 
I will bear the corresponding liabilities and consequences.

\vspace{\baselineskip}

%%%%%%%%%%%%%%%%%%%%%%%%%%%%%%%%%%%%%%%%%%%%%%%%%%%%%%%%%%%%%
% 填写说明:记得填写姓名和日期
%%%%%%%%%%%%%%%%%%%%%%%%%%%%%%%%%%%%%%%%%%%%%%%%%%%%%%%%%%%%%
\noindent
学生签字/Signature of Student:
$
\vcenter{\hbox{\includegraphics[height=0.5cm, keepaspectratio]{figures/your_signature.png}}}  % 插入手写签名图片
$
\hfill              
日期/Date: 20YY-MM-DD

\end{document}
